\pagenumbering{roman}
\setcounter{page}{1}

\selecthungarian

%----------------------------------------------------------------------------
% Abstract in Hungarian
%----------------------------------------------------------------------------
\chapter*{Kivonat}\addcontentsline{toc}{chapter}{Kivonat}

Az okosotthon rendszerek és otthon automatizálás technológia alapú megoldással nyújtanak kényelmi és egyéb szolgáltatásokat embereknek az otthoni környezetükben. Ez leginkább a ház "okos" eszközökkel (pl. villanykörték, hőmérséklet- és nedvességszenzorok, fűtési- és hűtési vezérlők) való felszerelését jelenti, hálózatba kötve egy vezérlő modullal, amit egy központi felhasználói felületen (általában egy okostelefonnal) lehet irányítani, akár a házon kívülről is. Ezek a rendszerek többek is lehetnek, mint egy kényelmi megoldás: növelhetik a ház energiahatékonyságát és bizonságát az erőforrások optimalizálásával, biztonsági felvételek analizálásával és egyéb módokon. Ez a szakdolgozat projekt bevezeti az olvasót az okosotthon rendszerek és otthon automatizálás világába, azoknak történelmébe, általános felépítésébe és kitér a piac vezető gyártóira, azoknak termékeire és megoldásaira. Ezt követően bemutatja egy fizikai demonstrációs modell tervezési és kivitelezési szakaszait, köztük a hardveres és szoftveres környezet kiválasztását. Végül bemutatja a demo modell használatát, felhasználási módjait és hiányosságait, továbbá a továbbfejlesztési lehetőségeket.

% A számítógépekkel vezérelt ipari rendszerek után az otthoni környezetben is felmerült ezeknek használata az otthonok energiatakarékosabbá és komfortosabbá tételéhez. Az ilyen rendszerekben a felhasználók a ház különböző rendszereit (pl. világítás, hűtés-fűtés, redőnyök) egy központosított felhasználói felületről vezérelhetik, valamint a mindennapi életet megkönnyítő automatizációkat állíthatnak be. A szakdolgozatomban bemutatom egy átlagos okosotthon rendszer felépítését, valamint elkészítek egy hasonló felépítésű, demonstrációs célú vezérelhető modellt és ennek a használati lehetőségeit kutatom. 
% \break
% % todo break fix, hogy ne ilyen magyar bekezdeses tragedia legyen

% Úgy vélem, hogy az okosotthon rendszerek használati lehetőségeinek kipróbálásához, egy valódi házban kialakított rendszer költséghatékony szimulálásához egy demonstrációs célú "doboz" modell megfelelőképpen alkalmas, ennek kialakítása során főképp két elvet követtem: a költséghatékonyságot (alacsony ár egy szakdolgozat projekthez) és a nyílt forráskódú szoftverek használatát. Manapság sajnos sokszor lehet hallani azt, hogy gyártók megszüntetik az egyébként nem annyira régi, még tökéletesen üzemelő termékeik támogatását, frissítését és ezzel sebezhetővé, nem működővé válnak, ami hozzájárul azok veszélyes hulladékká válásához. Ezt a problémát mérsékli az, ha valamilyen nyílt forráskódú szoftveres támogatás van azokra, mert a gyártás után évekkel később is friss szoftverrel tudnak üzemelni, illetve az általam kiválasztott hardverek később más projektekben is felhasználhatók. %todo forrás kivezetett termékre, pl. google nest secure
% \break

% A kereslet az okosotthon- és otthonautomatizációs technlólógiákra a megjelenésüktől kezdve folyamatosan növekedik, aktív piaci potenciál van bennük (a Világgazdasági Fórum (WEF) szerint 2030-ra elérheti a 13 billiót). Kezdetben főként csak egy kényelmi termékként gondoltak rájuk (például távolról való vezérelhetőség, egy központi felület), de emellé egyfajta megoldást adott a hatékonyságra és biztonságra is: előbbire például az automatizált fűtéssel, mozgásérzékelős lámpákkal, utóbbira a kamerákkal történő biztonsági felvételekkel, ennek elemzésével és a mozgáskorlátozottak támogatása.  %todo source from SHS article
\break

% A bevezető fejezetben szót ejtek az okosotthon megoldások történetéről, bemutatom egy átlagos okosotthon rendszer felépítését és a piacon elérhető elterjedtebb megoldásokat, valamint a témával való korábbi tapasztalataimról, motivációmról is írok. Az ezt követő tervezés fejezetben a demo modell tervezésével kapcsolatos előkészületeket mutatom be: annak általános céljait, követelményeit és elveit, továbbá a hardveres és szoftveres környezetek kiválasztását, annak indoklásait. A tervezést követően az elkészítés lépéseit mutatom be: a ház kialakítását, mikrokontrollerre történő eszközök bekötését, hálózat vitelét és ezeknek a szoftveres környezetének kialakítását. A modell elkészítését követően annak használatát mutatom be, értékelem az eredményeket és végül a továbbfejlesztési lehetőségeit ismertetem.


\vfill
\selectenglish


%----------------------------------------------------------------------------
% Abstract in English
%----------------------------------------------------------------------------
\chapter*{Abstract}\addcontentsline{toc}{chapter}{Abstract}

Smart home systems and home automation are technology based solutions for providing convenience and other services for people in their home environments. This is mostly in the form of a house fitted with "smart" devices (eg. lightbulbs, temperature and humidity sensors, heating and cooling actuators) networked with a controller module, which can be controlled via a user interface (generally via a smartphone), even remotely outside the house. These systems can be more, than a convenience solution: they can also increase the energy efficiency and safety of the house with resource usage optimization, analysis of security footage and other means. This thesis project introduces the reader to smart home systems and home automation, their history, general architecture and current market leading manufacturers, products and solutions. It is then followed by a presentation of a physical demonstrational model's planning and implementation phases, including the selection of hardware and software environments. Finally, showcases and evaluates the demo model's usage, use cases, shortcomings and provides opportunities of further development.

% After the usage of computer controlled systems in industrial installations, a demand for them appeared in home enviroments as well, in order to make homes more energy efficient and more comfortable to use. With these systems, users are able control different parts of their house (eg. lighting, cooling-heating, shutters) from a central user interface and set up automations to make their everyday life easier. In my thesis, I will be presenting the structure of a typical smart home system, furthermore create a physical model for demonstrational purposes and research its usage options.\break





% todo maybe summary of contents running text

\vfill
\selectthesislanguage

\newcounter{romanPage}
\setcounter{romanPage}{\value{page}}
\stepcounter{romanPage}
