\pagenumbering{roman}
\setcounter{page}{1}

\selecthungarian

%----------------------------------------------------------------------------
% Abstract in Hungarian
%----------------------------------------------------------------------------
\chapter*{Kivonat}\addcontentsline{toc}{chapter}{Kivonat}

A számítógépekkel vezérelt ipari rendszerek után az otthoni környezetben is felmerült ezeknek használata az otthonok energiatakarékosabbá és komfortosabbá tételéhez. Az ilyen rendszerekben a felhasználók a ház különböző rendszereit (pl. világítás, hűtés-fűtés, redőnyök) egy központosított felhasználói felületről vezérelhetik, valamint a mindennapi életet megkönnyítő automatizációkat állíthatnak be. A szakdolgozatomban bemutatom egy átlagos okosotthon rendszer felépítését, valamint elkészítek egy hasonló felépítésű, demonstrációs célú vezérelhető modellt és ennek a használati lehetőségeit kutatom.

TODO
introduction stuff: piaci megoldások, nyílt forráskódú
Philips hue

tervezés, milyen platform, miért arra esett a kiválasztás
elkészítés
használat, értékelés, elemzés, továbbfejlesztési lehetőségek


\vfill
\selectenglish


%----------------------------------------------------------------------------
% Abstract in English
%----------------------------------------------------------------------------
\chapter*{Abstract}\addcontentsline{toc}{chapter}{Abstract}

After the usage of computer controlled systems in industrial installations, a demand for them appeared in home enviroments as well, in order to make homes more energy efficient and more comfortable to use. With these systems, users are able control different parts of their house (eg. lighting, cooling-heating, shutters) from a central user interface and set up automations to make their everyday life easier. In my thesis, I will be presenting the structure of a typical smart home system, furthermore create a physical model for demonstrational purposes and research its usage options.

TODO

\vfill
\selectthesislanguage

\newcounter{romanPage}
\setcounter{romanPage}{\value{page}}
\stepcounter{romanPage}
