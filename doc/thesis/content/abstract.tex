\pagenumbering{roman}
\setcounter{page}{1}

\selecthungarian

%----------------------------------------------------------------------------
% Abstract in Hungarian
%----------------------------------------------------------------------------
\chapter*{Kivonat}\addcontentsline{toc}{chapter}{Kivonat}

Az okosotthon rendszerek és otthon automatizációk technológia alapú megoldással nyújtanak kényelmi és egyéb szolgáltatásokat embereknek az otthoni környezetükben. Ez leginkább a ház "okos" eszközökkel (pl. villanykörték, hőmérséklet- és nedvességszenzorok, fűtési- és hűtési vezérlők) való felszerelését jelenti, hálózatba kötve egy vezérlő modullal, amit egy központi felhasználói felületen (általában egy okostelefonnal) lehet irányítani, akár a házon kívülről is. Ezek a rendszerek többek is lehetnek, mint egy kényelmi megoldás: növelhetik a ház energiahatékonyságát és bizonságát az erőforrások optimalizálásával, biztonsági felvételek analizálásával és egyéb módokon. Ez a szakdolgozat projekt bevezeti az olvasót az okosotthon rendszerek és otthon automatizálás világába, azoknak történelmébe, általános felépítésébe és kitér a piac vezető gyártóira, azoknak termékeire és megoldásaira. Ezt követően bemutatja egy fizikai demonstrációs modell tervezési és kivitelezési szakaszait, köztük egy hardveres és szoftveres környezet kiválasztását. Végül bemutatja a demo modell használatát, felhasználási módjait és hiányosságait, továbbá annak továbbfejlesztési lehetőségeit.

\vfill
\selectenglish

%----------------------------------------------------------------------------
% Abstract in English
%----------------------------------------------------------------------------
\chapter*{Abstract}\addcontentsline{toc}{chapter}{Abstract}

Smart home systems and home automation are technology based solutions for providing convenience and other services for people in their home environments. This is mostly in the form of a house fitted with "smart" devices (eg. lightbulbs, temperature and humidity sensors, heating and cooling actuators) networked with a controller module, which can be controlled via a user interface (generally via a smartphone), even remotely outside the house. These systems can be more, than a convenience solution: they can also increase the energy efficiency and safety of the house with resource usage optimization, analysis of security footage and other means. This thesis project introduces the reader to smart home systems and home automation, their history, common architecture and current market leading manufacturers, their products and solutions. It is then followed by a presentation of a physical demonstrational model's planning and implementation phases, including the selection of a hardware and a software environment. Finally, it showcases and evaluates the demo model's general usage, use cases, shortcomings and provides opportunities of further development.

\vfill
\selectthesislanguage

\newcounter{romanPage}
\setcounter{romanPage}{\value{page}}
\stepcounter{romanPage}
