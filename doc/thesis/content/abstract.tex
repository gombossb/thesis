\pagenumbering{roman}
\setcounter{page}{1}

\selecthungarian

%----------------------------------------------------------------------------
% Abstract in Hungarian
%----------------------------------------------------------------------------
\chapter*{Kivonat}\addcontentsline{toc}{chapter}{Kivonat}

A számítógépekkel vezérelt ipari rendszerek után az otthoni környezetben is felmerült ezeknek használata az otthonok energiatakarékosabbá és komfortosabbá tételéhez. Az ilyen rendszerekben a felhasználók a ház különböző rendszereit (pl. világítás, hűtés-fűtés, redőnyök) egy központosított felhasználói felületről vezérelhetik, valamint a mindennapi életet megkönnyítő automatizációkat állíthatnak be. A szakdolgozatomban bemutatom egy átlagos okosotthon rendszer felépítését, valamint elkészítek egy hasonló felépítésű, demonstrációs célú vezérelhető modellt és ennek a használati lehetőségeit kutatom.
\newline
% todo break fix, hogy ne ilyen magyar bekezdeses tragedia legyen

Úgy vélem, hogy az okosotthon rendszerek használati lehetőségeinek kipróbálásához, egy valódi házban kialakított rendszer költséghatékony szimulálásához egy demonstrációs célú "doboz" modell megfelelőképpen alkalmas, ennek kialakítása során főképp két elvet követtem: a költséghatékonyságot és a nyílt forráskódú szoftverek használatát. Manapság sajnos sokszor lehet hallani azt, hogy gyártók megszüntetik az egyébként nem annyira régi, még tökéletesen üzemelő termékeik támogatását, frissítését és ezzel sebezhetővé, nem működővé válnak, ami hozzájárul azok veszélyes hulladékká válásához. Ezt a problémát mérsékli az, ha valamilyen nyílt forráskódú szoftveres támogatás van azokra, mert a gyártás után évekkel később is friss szoftverrel tudnak üzemelni, illetve az általam kiválasztott hardverek később más projektekben is felhasználhatók.
\newline

A bevezető fejezetben szót ejtek az okosotthon megoldások történetéről, bemutatom egy átlagos okosotthon rendszer felépítését és a piacon elérhető elterjedtebb megoldásokat, valamint a témával való korábbi tapasztalataimról, motivációmról is írok. Az ezt követő tervezés fejezetben a demo modell tervezésével kapcsolatos előkészületeket mutatom be: annak általános céljait, követelményeit és elveit, továbbá a hardveres és szoftveres környezetek kiválasztását, annak indoklásait. A tervezést követően az elkészítés lépéseit mutatom be: a ház kialakítását, mikrokontrollerre történő eszközök bekötését, hálózat vitelét és ezeknek a szoftveres környezetének kialakítását. A modell elkészítését követően annak használatát mutatom be, értékelem az eredményeket és végül a továbbfejlesztési lehetőségeit ismertetem.


\vfill
\selectenglish


%----------------------------------------------------------------------------
% Abstract in English
%----------------------------------------------------------------------------
\chapter*{Abstract}\addcontentsline{toc}{chapter}{Abstract}

After the usage of computer controlled systems in industrial installations, a demand for them appeared in home enviroments as well, in order to make homes more energy efficient and more comfortable to use. With these systems, users are able control different parts of their house (eg. lighting, cooling-heating, shutters) from a central user interface and set up automations to make their everyday life easier. In my thesis, I will be presenting the structure of a typical smart home system, furthermore create a physical model for demonstrational purposes and research its usage options.

TODO

\vfill
\selectthesislanguage

\newcounter{romanPage}
\setcounter{romanPage}{\value{page}}
\stepcounter{romanPage}
