\pagenumbering{roman}
\setcounter{page}{1}

\selecthungarian

%----------------------------------------------------------------------------
% Abstract in Hungarian
%----------------------------------------------------------------------------
\chapter*{Kivonat}\addcontentsline{toc}{chapter}{Kivonat}

A számítógépekkel vezérelt ipari rendszerek után az otthoni környezetben is felmerült ezeknek használata az otthonok energiatakarékosabbá és komfortosabbá tételéhez. Az ilyen rendszerekben a felhasználók a ház különböző rendszereit (pl. világítás, hűtés-fűtés, redőnyök) egy központosított felhasználói felületről vezérelhetik, valamint a mindennapi életet megkönnyítő automatizációkat állíthatnak be. A szakdolgozatomban bemutatom egy átlagos okosotthon rendszer felépítését, valamint elkészítek egy hasonló felépítésű, demonstrációs célú vezérelhető modellt és ennek a használati lehetőségeit kutatom. 
\break
% todo break fix, hogy ne ilyen magyar bekezdeses tragedia legyen

Úgy vélem, hogy az okosotthon rendszerek használati lehetőségeinek kipróbálásához, egy valódi házban kialakított rendszer költséghatékony szimulálásához egy demonstrációs célú "doboz" modell megfelelőképpen alkalmas, ennek kialakítása során főképp két elvet követtem: a költséghatékonyságot (alacsony ár egy szakdolgozat projekthez) és a nyílt forráskódú szoftverek használatát. Manapság sajnos sokszor lehet hallani azt, hogy gyártók megszüntetik az egyébként nem annyira régi, még tökéletesen üzemelő termékeik támogatását, frissítését és ezzel sebezhetővé, nem működővé válnak, ami hozzájárul azok veszélyes hulladékká válásához. Ezt a problémát mérsékli az, ha valamilyen nyílt forráskódú szoftveres támogatás van azokra, mert a gyártás után évekkel később is friss szoftverrel tudnak üzemelni, illetve az általam kiválasztott hardverek később más projektekben is felhasználhatók. %todo forrás kivezetett termékre, pl. google nest secure
\break

A kereslet az okosotthon- és otthonautomatizációs technlólógiákra a megjelenésüktől kezdve folyamatosan növekedik, aktív piaci potenciál van bennük (a Világgazdasági Fórum (WEF) szerint 2030-ra elérheti a 13 billiót). Kezdetben főként csak egy kényelmi termékként gondoltak rájuk (például távolról való vezérelhetőség, egy központi felület), de emellé egyfajta megoldást adott a hatékonyságra és biztonságra is: előbbire például az automatizált fűtéssel, mozgásérzékelős lámpákkal, utóbbira a kamerákkal történő biztonsági felvételekkel, ennek elemzésével és a mozgáskorlátozottak támogatása.  %todo source from SHS article
\break

% A bevezető fejezetben szót ejtek az okosotthon megoldások történetéről, bemutatom egy átlagos okosotthon rendszer felépítését és a piacon elérhető elterjedtebb megoldásokat, valamint a témával való korábbi tapasztalataimról, motivációmról is írok. Az ezt követő tervezés fejezetben a demo modell tervezésével kapcsolatos előkészületeket mutatom be: annak általános céljait, követelményeit és elveit, továbbá a hardveres és szoftveres környezetek kiválasztását, annak indoklásait. A tervezést követően az elkészítés lépéseit mutatom be: a ház kialakítását, mikrokontrollerre történő eszközök bekötését, hálózat vitelét és ezeknek a szoftveres környezetének kialakítását. A modell elkészítését követően annak használatát mutatom be, értékelem az eredményeket és végül a továbbfejlesztési lehetőségeit ismertetem.


\vfill
\selectenglish


%----------------------------------------------------------------------------
% Abstract in English
%----------------------------------------------------------------------------
\chapter*{Abstract}\addcontentsline{toc}{chapter}{Abstract}

After the usage of computer controlled systems in industrial installations, a demand for them appeared in home enviroments as well, in order to make homes more energy efficient and more comfortable to use. With these systems, users are able control different parts of their house (eg. lighting, cooling-heating, shutters) from a central user interface and set up automations to make their everyday life easier. In my thesis, I will be presenting the structure of a typical smart home system, furthermore create a physical model for demonstrational purposes and research its usage options.\break

I believe that a demonstrational "box" model is an adequate and cost-effective solution to try out the usage of smart home systems and to simulate a proper system installed in a real house. During its implementation phase, I tried to follow two basic principles: cost-effectiveness (cheap price for a thesis project) and the usage of open-source software. We can often hear these days, that companies end supporting and updating their not so old, still completely functional products, which make them vulnerable to attacks and contributes to them becoming hazardous waste. This problem can be mitigated to some extent with the usage of actively maintained open source software available to such devices, because this way they can run up-to-date software years after manufacturing, furthermore the devices chosen by me can be later repurposed in other projects.\break

The demand for smart home system and home automation technology has been steadily increasing since their inception and there is active market potential in them (the Worldwide Economic Forum predicts it could react 13 trillion dollars by 2030). Initially, they were considered only as a convenience product (with remote control, a centralized control interface), but since then they have also become a solution of efficiency and safety. Examples for the earlier are automatized heating and motion controlled lights, for the latter are recorded security footage and its analysis, support of handicapped people.

% todo maybe summary of contents running text

\vfill
\selectthesislanguage

\newcounter{romanPage}
\setcounter{romanPage}{\value{page}}
\stepcounter{romanPage}
