%----------------------------------------------------------------------------
\chapter{\bevezetes}
%----------------------------------------------------------------------------

% A bevezető tartalmazza a diplomaterv-kiírás elemzését, történelmi előzményeit, a feladat indokoltságát (a motiváció leírását), az eddigi megoldásokat, és ennek tükrében a hallgató megoldásának összefoglalását.

% A bevezető szokás szerint a diplomaterv felépítésével záródik, azaz annak rövid leírásával, hogy melyik fejezet mivel foglalkozik.

In most conventional, "non-smart" houses or flats, utility and convenience devices (eg. lighting, heating, ventilation) are mostly controlled manually by hand via simple switches, knobs or the combination of the two, which requires people to be physically present at the scene to actuate them. In a home equipped with a smart home system, an extra computer and relay or transistor based system is added to them, which assists their actuation from centralized user interface (most often controlled on a smartphone application). Smart home systems can also be extended with digital sensors and other more fine tunable, convenience and security devices (in general just called as "smart" devices), such as temperature and humidity sensors, RGB lights, keycard readers, cameras. The sensors, along with other kinds of input (eg. time of the day, phone sensing) can be used to automate processes, which in turn would require manual interaction, such as turning on or off lights in rooms based on presence, rolling a shutter on or off at a certain time of a day. The general term for this process is called home automation, but this term also means the process of installation and usage a smart home system, so their meaning is almost identical in most contexts. \break

I believe that a demonstrational "box" model is an adequate and cost-effective solution to try out the usage of smart home systems and to simulate a proper system installed in a real house. During its implementation phase, I tried to follow two basic principles: cost-effectiveness (cheap price for a thesis project) and the usage of open-source software. We can often hear these days, that companies end supporting and updating their not so old, still completely functional products, which make them vulnerable to attacks and contributes to them becoming hazardous waste. This problem can be mitigated to some extent with the usage of actively maintained open source software available to such devices, because this way they can run up-to-date software years after manufacturing, furthermore the devices chosen for the thesis project can be later repurposed in other projects.\break

The demand for smart home system and home automation technology has been steadily increasing since their inception and there is active market potential in them (the Worldwide Economic Forum predicts it could react 13 trillion dollars by 2030). Initially, they were considered only as a convenience product (with remote control, a centralized control interface), but since then they have also become a solution of efficiency and safety. Examples for the earlier are automatized heating and motion controlled lights, for the latter are recorded security footage and its analysis, support of handicapped people.
% todo source

% sw is just as much important as hw
% connection to iot

% TODO
\section{Motivation, previous work}
My motivation behind choosing this thesis project was mostly my previous work on the topic and also that it combines more fields of expertise: various fields in Information Technology along with basic electronics.\break

For Training Project Laboratory, me and my two fellow students used an ESP-based microcontroller to control a few small electric components connected to it, which in large resembled a house's utilities. However, this was only a very rudimentary breadboard model, with continuous debug printing to the serial console and only a potentiometer and two buttons as inputs, but it was useful to familiarize ourselves with the usage and programming of a modern microcontroller liked by many hobby enthusiasts.\break
% todo maybe pic

Throughout Project Laboratory, I took the project further and made a proper shoebox smart home system model. Along with a basic floorplan for various rooms and electronic components placed in them, it featured a web-controllable user interface. Its software architecture was comprised of three main components: a JavaScript backend responsible for maintaining the components' status and communication between the microcontroller and user, microcontroller code for controlling the electronic components (reading sensor data, setting output for actuators), receiving and sending data to the backend and finally a React-based frontend to send commands and receive readings to and from the backend. \break
%I decided to apply with it for Gadget Competition 2024 (organized by BME-MIT) and achieved 4th place on the competition.
% todo maybe pic of box and ui

For the Thesis Project, I've decided to take the shoebox model further, focus more on the software side of improvement - the hardware, electronic components inside the box have remained mostly the same. I believe that the software of my Project Laboratory project was sufficient for showcasing simple smart home use cases in a demo environment, however would have been harder to further develop to extend with more devices and features than using already existing open-source smart home software with good community support, which I had also been looking forward to try out.

\section{A brief history of smart home systems}
- precursor: control of industrial systems via electric, then electronic machines
- first solutions
- development of protocols (not only used for smart homes) wifi, iot, zigbee, z-wave
- cloud, subscription world and future, why local hosting is important
\section{Stucture of a typical smart home system}
a typical smart home system's structure
- controller, maybe cloud
- devices: actuators, sensors, etc.
- user interfaces, protocols
- etc, tadada
\section{Current solutions of the market}
- products
- alexa, google voice, etc voice assistants
