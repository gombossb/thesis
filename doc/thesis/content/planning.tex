\chapter{Planning}

The smart home model box's development and planning began during Project Laboratory and has continued since then with some variations, including the replacement of software platform. This chapter takes a look at the planning work, before the plans were put into practice.

\section{General goals, requirements and principals of the planned model}

The box project's goal was to get familiar with the most important IoT and related technologies, such as microcontrollers, networking and electronics control and to put this knowledge into practice via the creation of a smart home model system demo box, which can then be used to gain experience with the usage of such a smart home or home automation system's evaluation.

During the box model project's planning and implementation phases, gaining a deeper insight of the underlying technologies and building a smart house solution from inexpensive, easy-to-obtain generic electronic components, microcontrollers and open-source software were favored over expensive, locked down out-of-the box solutions, as they are tuned for a quick and optimal end-user experience, but hide the implementation complexity from their users, therefore are less suitable for the project. The utilized software platform should be self-hosted in order to ensure absolute ownership of data and mitigate the vulnerability of being exposed to an external provider.

When finished, the box model should have electric and electronic devices inside it, which resemble utilities, that can be controlled along with sensors, detectors that collect the momentary status of different natural aspects at a given interval. It should also have a user interface, which can be used to control the appliances, display the current and historic readings of sensors accessible from a web browser or smartphone.

\section{Hardware environment}

The box containing the project's internals was chosen to be a shoebox, which I already had at hand and and is large enough to have separated rooms and smaller electric and electronic components inside it, but is also small enough to be easily transported. The floorplan of the house wasn't fixed at the planning phase due to the uncertainty of the picked devices' sizes, it was structured later in the implementation phase with cardboard.
% floorplan

As for electronics inside the box, I decided on a single microcontroller board setup, where it is used to either control or read input from the connected devices and communicate to a server, which maintains all devices' status and provides a user interface. A microcontroller board is essentially a small computer, which is designed in a way to manage specific tasks within an embedded system without requiring a complex operating system. \cite{IBMmicrocontroller} Microcontrollers are well-suited for applications requiring real-time signal processing, such as controlling motors and servos and interfacing with various types of sensors and communications, therefore a great platform for home automation applications. The major microcontroller platforms considered for the project were Arduino, ESP and STM32. During Training Project Laboratory, I had the opportunity to use a NodeMCU ESP32S microcontroller, which is based on the ESP32S chip, that besides having a typical microcontroller functionality (ADC and DAC, PWM, UART and many more via their pins), also has wireless communication capabilities via Bluetooth and Wi-Fi. Due to this extra functionality often not being integrated in other microcontrollers, the cheap price and better price-to-value ratio, as well as the previous experience and easy access, I chose a NodeMCU ESP32S microcontroller board for the project. \cite{ESPvArduino} \cite{ESPvSTM}


LEDs, cables, resistors, transistors, ldr, servo, peltier, breadboard

- networking - tp link soho router at hand, esp supports wifi, extra wifi adapter

\section{Software environment}

mention architecture of project laboratory
usage of open source software already available, modular, extensible
- software - homeassistant, esphome: popular open source projects, good community support, many features
other projects?

- - easy to set up

- - yaml files
